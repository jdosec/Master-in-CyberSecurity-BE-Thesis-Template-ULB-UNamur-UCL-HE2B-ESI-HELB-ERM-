\chapter{Literature review, state of the art (SotA), definitions and notations}
\section{Divide and conquer: From title to sub-questions}
% split the main problem into sub-problems, show how related they are and then explore each with state of the art

\section{Notations}
\subsection{Example of potential boxes to make a \LaTeX{} env from}
\begin{bclogo}[arrondi=0.1, logo=\bcquestion, couleur=grey,noborder=true]{Open question}
    \lipsum[1-1]
\end{bclogo}


\begin{bclogo}[arrondi=0.1, logo=\bcattention, couleur=grey,noborder=true]{Important remark}
    \lipsum[1-1]
\end{bclogo}

\begin{bclogo}[arrondi=0.1, logo=\bcpanchant, couleur=grey,noborder=true]{Restrictions, limitations, work in progress}
    \lipsum[1-1]
\end{bclogo}

\begin{bclogo}[arrondi=0.1, logo=\bcinfo, couleur=grey,noborder=true]{Reminder}
    \lipsum[1-1]
\end{bclogo}


\begin{bclogo}[arrondi=0.1, logo=\bclampe, couleur=grey,noborder=true]{idea/opportunity/contribution/future work}
    \lipsum[1-1]
\end{bclogo}

\begin{bclogo}[arrondi=0.1, logo=\bccrayon, couleur=grey,noborder=true]{Side note, personal thought}
    \lipsum[1-1]
\end{bclogo}

\begin{bclogo}[arrondi=0.1, logo=\bccle, couleur=grey,noborder=true]{key element to understand section or to remember from this section}
    \lipsum[1-1]
\end{bclogo}

\begin{bclogo}[arrondi=0.1, logo=\bcbombe, couleur=grey,noborder=true]{warning}
    \lipsum[1-1]
\end{bclogo}

\begin{bclogo}[arrondi=0.1, logo=\bcbook, couleur=grey,noborder=true]{definition}
    \lipsum[1-1]
\end{bclogo}

\section{Definitions}%where did you find these, based on what criterii, why would that one be the most suitable, etc.
\subsection{List of definitions}
\subsection{Summary of the relationships between the defined concepts}
\section{State of the art and related works}
\subsection{Publications}% "peer-reviewed" publication (double blinded when possible) are the core. Other types are "informational"
\subsubsection{Review methodology}% You are exploring & comparing others work. What makes your results valid, relevant, etc.? What are your metrics? How can you ensure you explored all that needs to be explored? What is you experimental environment and methodology to measure and compare the tools' performance?
\subsection{Implementations, standards, protocols, technologies, ...}
\subsubsection{Review methodology}% You are exploring & comparing others work. What makes your results valid, relevant, etc.? What are your metrics? How can you ensure you explored all that needs to be explored? What is you experimental environment and methodology to measure and compare the tools' performance?
\subsection{Summary}% Consider including a table/visual representation to easily grasp the section
\pagestyle{fancy}
%\section{This is the first section}
\lipsum[1-5] 
\gls{energyconsump} is the power consumption and is defined w.r.t.\ the energy consumption \gls{energyproduction} over a time $\Delta t$ \eqref{eq:eq2}.
\begin{align}
	\gls{energyconsump} = \frac{\gls{energyproduction}}{\Delta t} \equnit{\si[per-mode=symbol]{\joule\per\second}}\label{eq:eq2}
\end{align}
\lipsum[1-15]
\cite{EncycloBritannica:ScientificMethod, wiki:ScientificMethod, wiki:MeetingsBloodyMeetings, schwaberSutherland2017ScrumGuide, khanacademyScientificMethod, melotElementRedactionSci2011, Kerckhoffs}

\chapter{Project's mission, objectives and requirements}
\lipsum[1-2]
% definition of this project's target
% what do we need?
% what does already exists (see previous chapter)?
% what does not yet exist?
% what shall we try to make exist?
% what will remain to do after that?

\section{Requirements}
\lipsum[1-2]
\subsection{List of requirements and how they relate one to another}
\subsection{Requirements covered by state of the art}
\subsection{Requirements not covered by state of the art}

\section{Project scoping}
\lipsum[1-2]
\subsection{Mission statement of this project}
\subsection{Explicit out-of-scope definition}
% what we will not do: because because x, y, z