\chapter{Initial project proposal}
% This section is the document you prepare before reaching/looking for a supervisor. It is, among other things, what helps your supervisor to decide to accept your or not within the research lab and supervise you.
\section{Title}
Title: ...
\section{Proposal summary}
\subsection{Participants}
Project Director/Owner: ...

Researchers(s):
\begin{itemize}
    \item a
    \item b
\end{itemize}

\subsection{Background}
% context, the situation, the problem, ...
\subsection{Objectives}
The goal of this project is to develop ...
% this should do this, that and that
\subsection{Expected outcome}
The project will produce a number of deliverables:
\begin{itemize}
    \item ...
    \item implementation that does x under conditions y ...
    \item a master thesis ...
    \item a draft of a white paper that can be further modified and improved upon for the research lab to consider publishing at a later date (with or without the student's further collaboration, upon his preference)
\end{itemize}

\section{Proposal description (max. 4 pages, ref's included)}
\subsection{Aim of the study and relevance for designated target group}
\subsection{State of the art}
% Describe the state of the art related to this proposal at a national and international level
\subsection{Global research context}
% In which global research context is this proposal situated
\subsection{Research strategy}
% Describe the methods, techniques and procedures by which the research will be conducted
\subsection{Collaboration}
\subsection{Expected outcome}
\subsection{Feasibility \& risks}
% Make your S.W.O.T. analysis. What are your competencies, resources, available time, etc.
\subsection{Yet another section (you can add your own too of course)}
\subsection{References}
% Be careful using the references appropriately

\section{Phasing of the project (max. 4 pages)}

Start date:

End date:

% Here is an example of starting generic workpackages and objectives you might have (you should certainly split it more based on your project)
% # Starting work packages
% Each project has typically a minimum of 5 starting work packages:

% ## Integrate research group
% ### Objective 0: Register/Access communication channels and study its content
% ### Objective 1: Plan/schedule your project to integrate the group (meetings, etc.), its processes and objectives
% ### Objective 2: Check/Act (from PDCA) Reviewing and corrective process
% * Includes presenting your outputs, processes, results and getting it validated (feedback and your modifications)

% ## TechnologyIntelligence
% Start global project shared documentation, technology intelligence (veille technologique, https://en.wikipedia.org/wiki/Technology_intelligence) receptacle & procedures. Outputs data (keywords, authors, venues, ...) as well as processes (code, systems, ...) to monitor and integrate these.
% ### Objective 0: Enumerate & document relevant communication channels and setup/program/code automated tools to monitor them
% * RSS on dblp for example https://dblp.org/faq/How+can+I+fetch+all+publications+of+one+specific+author.html
% * auto import/update from RSS to zotero/mendley group with tagging
% * gits with continuous integration (submodules) and archiving
% * Google Alerts, ...
% ### Objective 1: Explore channels and discover/document/program relevant parameters (keywords, labels, product names, authors, venues, notions, tags, categories, ...).
% Consider generic top tiers cybersecurity venues + venues very specific to your topic
% * Generic CyberSecurity Venues
% 	* https://people.engr.tamu.edu/guofei/sec_conf_stat.htm ![image](uploads/08fc5d627c9b4da024e53a365d4927ae/image.png)
% 	* http://jianying.space/conference-ranking.html ![image](uploads/21d12f24d588107d088c707df3443e27/image.png)
% * Specific to cyber range: International Workshop on Cyber Range Technologies and Applications (CACOE 2020) https://ieeexplore.ieee.org/document/9229838/metrics#metrics
% ### Objective 2: Start your intelligence analysis process (stack the papers, rank them, etc.)
% ### Objective 3: Check/Act (from PDCA) Reviewing and corrective process
% * Includes presenting your outputs, processes, results and getting it validated (feedback and your modifications)

% ## SotA : State of the art
% ### Objective 0: Enumeration of technologies, tools, authors, venues, vendors, ... (input data)
% ### Objective 1: Enumeration & analysis of features, technologies/techniques, pros, cons, restrictions, ... (input data)
% ### Objective 2: Enumeration of project requirements (desired output)
% ### Objective 3: Classification of features & requirements in core/essentials/mandatory, desired, "nice to have", neutral, against (desired output)
% ### Objective 4: Matching/analysis of input versus desired output
% One part will be "this can be reached with that, that and that combined in this way". And another, the reminder, should be "these requirements cannot be satisfied yet with what exist" = your target contribution.
% ### Objective 5: Thesis chapter
% ### Objective 6: Learning process / Documentation chapter
% * For those who are joining the project (they shouldn't need to explore, give them the straight line to what is required to read, learn, experiment)
% ### Objective 7: Check/Act (from PDCA) Reviewing and corrective process
% * Includes presenting your outputs, processes, results and getting it validated (feedback and your modifications)

% ## Roadmap
% Canvassing the project in any foreseeable work packages. Typically, not all will be implemented by the same person. Allows new contributors to join and contribute.
% ### Objective 1: Project's Big Bang: initial burst of creation of packages etc. Enumerate, design, write down, discuss all the work packages, objectives and tasks that you can think of.
% ### Objective 2: Continuous refactoring, updating and integration of new work packages, objectives and tasks.
% ### Objective 3: Check/Act (from PDCA) Reviewing and corrective process
% * Includes presenting your outputs, processes, results and getting it validated (feedback and your modifications)

% ## Prototype
% Create the first prototype using the tools and objectives we have [misc.conventions.objectivestools](misc.conventions.objectivestools)
% ### Objective 0: Establish/Start P.D.C.A. cycle (define objectives, tasks, KPI, ...)
% ### Objective 1: Obtain skills, knowledge and setup labs & other necessary requirements
% ### Objective 2: Gather initial/foreseen requirements
% ### Objective 3: Design testing
% ### Objective 4: Build & test first running prototype.
% Priorities are in this order:
% 1. Make it work (should do what is expected). Deliverable first! If it's "half done" it's not done. If it's easier/faster for someone else with lots of experience to start the project from scratch ... then you have made no contribution.
% 2. Make it usable (it should be easy to use). (same remark as previous)
% 3. Make it nice (the code should be easy to maintain, documented, etc.) (same remark as previous)
% 3. Make it secure
% 4. Make it private
% 5. Make it fast (efficient, does not use many resources)
% ### Objective 5: Program running prototype for the various deployment mode
% see [misc.conventions.objectivestools](misc.conventions.objectivestools)
% ### Objective 6: Continuous evaluation and improvement security
% ### Objective 7: Continuous evaluation and improvement of privacy
% ### Objective 8: Continuous evaluation and improvement with regard to mandatory frameworks (if any)
% ### Objective 9: Continuous integration with the other work packages, projects etc. in the lab
% ### Objective 10: Write corresponding Thesis chapters and documentation
% ### Objective 11: Finish P.D.C.A. cycle (evaluate)

% ## Publication & peer review (documentation, thesis, ...)
% ### Objective 1: Learn, analyse & discuss expectations, rules, evaluation grid
% * Regarding: thesis, documentation, white paper
% * Checkout templates (https://www.overleaf.com/latex/templates/master-in-cybersecurity-be-thesis-template-ulb-unamur-ucl-he2b-slash-esi-helb-erm/ypmhcxmmtgkn), guidelines, evaluation grids, ...

% ### Objective 2: Write and review
% * For each #DocumentTypeToProduce
% 	* For each #Chapter to write do PDCA
% 		* Plan
% 			* Structure the corresponding chapter (train of thoughts, key arguments, liaison, etc. check McGarrity's introduction to public speaking)
% 		* Do
% 			* Fill in the key points in each section (in comments)
% 			* Write properly the first finished version
% 		* Check
% 			* Present results for review (demo, submit for review by whole lab, discuss)
% 		* Act
% 			* Make modifications and document/keep track/document suggestions with examples to share with others
% ### Objective 3: Check & plan publication
% * For each type
% * Check the venue, the conditions
% 	* This includes: what are the requirement to publish your master thesis, deadlines, side, length, layout, format, ... where you need to send to, to whom, etc.

\subsection{Workpackage 1: title}
Start date:

End date:
\subsubsection{Description}
\subsubsection{S.M.A.R.T. Objectives}
% List S.M.A.R.T. objectives https://en.wikipedia.org/wiki/SMART_criteria
\subsubsection{Deliverables and their K.P.I.s}
% Deliverables and how to evaluate them https://en.wikipedia.org/wiki/Performance_indicator
\subsubsection{Participants and responsibility assignment matrix (RACI model)}
% https://en.wikipedia.org/wiki/Responsibility_assignment_matrix
\subsection{Workpackage 2: title}
Start date:

End date:
\subsubsection{Description}
\subsubsection{S.M.A.R.T. Objectives}
% List S.M.A.R.T. objectives https://en.wikipedia.org/wiki/SMART_criteria
\subsubsection{Deliverables and their K.P.I.s}
% Deliverables and how to evaluate them https://en.wikipedia.org/wiki/Performance_indicator
\subsubsection{Participants and responsibility assignment matrix (RACI model)}
% https://en.wikipedia.org/wiki/Responsibility_assignment_matrix
\subsection{Workpackage 3: title}
Start date:

End date:
\subsubsection{Description}
\subsubsection{S.M.A.R.T. Objectives}
% List S.M.A.R.T. objectives https://en.wikipedia.org/wiki/SMART_criteria
\subsubsection{Deliverables and their K.P.I.s}
% Deliverables and how to evaluate them https://en.wikipedia.org/wiki/Performance_indicator
\subsubsection{Participants and responsibility assignment matrix (RACI model)}
% https://en.wikipedia.org/wiki/Responsibility_assignment_matrix
\subsection{Workpackage 4: title}
Start date:

End date:
\subsubsection{Description}
\subsubsection{S.M.A.R.T. Objectives}
% List S.M.A.R.T. objectives https://en.wikipedia.org/wiki/SMART_criteria
\subsubsection{Deliverables and their K.P.I.s}
% Deliverables and how to evaluate them https://en.wikipedia.org/wiki/Performance_indicator
\subsubsection{Participants and responsibility assignment matrix (RACI model)}
% https://en.wikipedia.org/wiki/Responsibility_assignment_matrix


\section{Expertise of the project's research team (max. 2 pages)}
\subsection{Expertise}
% Describe yours and the expertise available to you for this project proposal
\subsection{Publications \& porfolio relevant to the project proposal}
% List of successes, projects, codes, publications relevant to show your ability to succeed in this endeavour

\section{Requirement (equipment, skills, ...)}
% Indicate what is required, how you plan to acquire it, etc. If you need to study an online course first, etc.

\section{Proposal summary table (max. 2 pages)}
% +/- 1 line per field
Domain:

Study director:

Research unit/staff/dept:

Title:

Aim of the study:

Research strategy:

Innovative character:

Target group \& relevance for that audience:
Partnerships:

\section{Evaluation \& Self-Evaluation}
\subsection{Self-Evaluation}
\begin{itemize}
    \item ... /10: Relevance and scope : Is the relevance of the proposed research well defined? Is the scope of project well described and delimited?
    \item ... /10: Efficiency : Are the required/allocated means means, i.e. personnel, infrastructure and equipment, consistent with the projected outcome of the project?
    \item ... /10: State of the Art : How is the state of the art, related to this proposal at a national and international level, described? Are the proposers aware of past and current similar research activities?
    \item ... /10: Research Strategy and Phasing : Are the phases of the project realistic, coherent and in line with the intended objectives? How realistic and well argumented are the objectives? Is the research team well organized and able to perform the planned research
    \item Remarks: 
\end{itemize}

\subsection{Evaluation}
\begin{itemize}
    \item ... /10: Relevance and scope
    \item ... /10: Efficiency
    \item ... /10: State of the Art
    \item ... /10: Research Strategy and Phasing
    \item Remarks: 
\end{itemize}