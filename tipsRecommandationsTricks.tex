\chapter{Tips, recommendations, useful references \& tricks}
Don't forget: "Scientia vincere tenebras"  so explore these with constructive and useful criticism
\begin{itemize}
    \item \url{https://actus.ulb.be/fr/presse/kit-media/le-principe-du-libre-examen}
    \item \url{https://www.belspo.be/belspo/organisation/publ/pub_ostc/Eth_code/ethcode_fr.pdf}
    \item Check: (FR) \url{http://informatique.umons.ac.be/algo/redacSci.pdf}
    \item Check: (FR) \url{https://www.fp.ulaval.ca/88clefs/francais/ex8b_02.html}
    \item Check: (EN) \url{https://www.fp.ulaval.ca/88clefs/anglais/index.html} (you can invite your reviewer to use this grid when commenting your text. It is extra work for them, but makes it easier for the both of you to communicate as you are sharing a single explicit dictionnary)
    \item Before reaching out to a supervisor, you should have prepared your IPP (see appendix)
    \item This template is light-years away from perfect. Do send me your suggestions for improvements :)
    \item Latex: Make one chapter per file
    \item Review: Overleaf allows review & commenting per "line" not per word. Therefore, to keep the review understandable (being able to see what the "comment" is about, keep one sentence per line.
    \item Structure: Consider the following and check best practices for each section (e.g. "apply recursively (to chapter, sections, ...): state it, explain it, conclude it", "have roper transition between chapters, sections, paragraphs, ...", "conclusion shouldn't introduce new information", etc.)
    \item \url{https://uh.edu/~lsong5/documents/A%20sample%20proposal%20with%20comment.pdf}
    \item \url{https://www.researchgate.net/publication/49704298_The_Art_of_Writing_Good_Research_Proposals}
    \item Grammar & spelling: before review, document shouldn't contain any spelling/grammar mistake that would be discovered by an automatic tool (example: \url{https://www.polishmywriting.com/}, \url{https://languagetool.org/}, ...)
    \item Check online for checklists, tips, etc. to proofread what you have done all along the way. If your supervisor spends his/her focus on correcting things you could have fixed by yourself before submitting for review, it's time & focus not spent on other things that are harder for you to find, understand & fix. Example, fallacies, assertion without demonstration, citation by simple "assert x"[18] (instead of stating who, what, how, why and how it relates to your own claims), ...
    \item Check: list of fallacies
    \item Check: the evaluation grid and the list of skills "référentiel de compétence" for the studies you are doing. This is essentially what you are demonstrating with this academic exercize. Check the Master thesis evaluation forms / grids you can find online. Your supervisor might not be at liberty to distribute the ones they are using, but this should help you understand what is expected of you.
    \item Check: \url{https://studylib.net/doc/7753735/checklist-for-revising-an-argument}
    \item Search online for best scientific practices: do discover/study/update your knowledge of best practices in scientific writing
    \item any figure should be numbered, labeled, have a caption and be mentioned in the text (in Fig. 2, you can see that etc.) If it does not, either it's missing or the figure has no use and should not be there.
    \item apply best software development practices (no dead code -> no dead text, figure, etc.), DRY, no copy/paste (if you mentioned something in your intro, conclusion and core, don't copy paste it. It's repeated but with different intentions. And therefore, should be formulated in such a fashion to accomplish that section's purpose. When you copy/paste, the reader's reaction will probably be "great, I just lost 10min of my life reading the same text again while trying to spot the new information"
    \item Use layout to make your text easy to read but make sure it can be read when printed in black \& white. You can separate the core of the dissertation from personal thoughts, theorems, definitions, etc. using boxes, footnotes, etc. See bclogo \url{http://www.tug.org/texlive/Contents/live/texmf-dist/doc/latex/crossword/cwpuzzle.pdf} with bclogos \url{http://tug.ctan.org/info/symbols/comprehensive/symbols-letter.pdf} and/or tcolorbox \url{https://www.overleaf.com/latex/examples/drawing-coloured-boxes-using-tcolorbox/pvknncpjyfbp}
    \item Consider using 1. a bibliography manager (Zotero, jabref, Mendeley) and 2. importing your bibtex entries directly from existing publisher/editor/author supported databases (such as DBLP or the article's publisher's website such as ieee, springer, elsevier, iacr, ...). Note that Wikipedia has a link on the left to proper citations formats. So it's really bad when one sees a document that does not even have proper bibliographic entries for wikipedia pages as each of these page's bibliographic entries are available on wikipedia itself.
    \item Burden of proof: Assertions need to be supported by your own argumentation or external ones. It is not sufficient to simply "quote" or "cite" that reference and assume it's the reader's responsibility to read the reference, study it, study your document and search for a way to see the relationship between the two and how the reference could be interpreted to support your argument. That responsibility lies within the one who assert/claims something. So when using external reference, explain who, what, why, how, when of the paper and how it supports your claim. For more, see also \url{https://lsa.umich.edu/sweetland/undergraduates/writing-guides/how-do-i-effectively-integrate-textual-evidence-.html}, \url{https://opentextbc.ca/writingforsuccess/chapter/chapter-9-citations-and-referencing/} \url{https://www.coursera.org/learn/public-speaking}
    \item Wikipedia contributors. (2021, April 19). Burden of proof (philosophy). In Wikipedia, The Free Encyclopedia. Retrieved 08:57, April 27, 2021, from \url{https://en.wikipedia.org/w/index.php?title=Burden_of_proof_(philosophy)&oldid=1018775341}
    \item consider splitting your pdf in two or three pdfs when sending them. One being the core, the second being the "optional reading" part (bibliography, appendices, ...). That way, it's easier to gauge the size of the work for the reader, it's easier to print, etc.
    \item consider sending your pdf in multiple "reading" format (prepare already the version to read on tablet, to print 2 pages per A4 sheet, etc.). I'm sure your reviewer will appreciate instead of being stuck with a pdf with large margins.
    \item consider sending your slideshow in advance, sending your sources (latex, etc.) before the presentation so that the members of your jury have the liberty to process it any way they want (they might be more comfortable with annotating x, y, z, or would like to perform static code analysis on your source code, etc.)
\end{itemize}
